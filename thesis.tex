%# -*- coding: utf-8-unix -*-
%%==================================================
%% thesis.tex
%%==================================================

% 双面打印
%\documentclass[doctor, fontset=adobe, openright, twoside, zihao=-4]{sjtuthesis}
\documentclass[bachelor, fontset=adobe, openany, oneside, zihao=5, submit]{sjtuthesis}
% \documentclass[master, adobefonts, review]{sjtuthesis}
% \documentclass[%
%   bachelor|master|doctor,	% 必选项
%   fontset=windows|adobe,  	% 只测试了adobe
%   oneside|twoside,		% 单面打印,双面打印(奇偶页交换页边距,默认)
%   openany|openright, 		% 可以在奇数或者偶数页开新章|只在奇数页开新章(默认)
%   zihao=-4|5,, 		% 正文字号:小四、五号(默认)
%   review,	 		% 盲审论文,隐去作者姓名、学号、导师姓名、致谢、发表论文和参与的项目
%   submit			% 定稿提交的论文,插入签名扫描版的原创性声明、授权声明
% ]

% 逐个导入参考文献数据库
\addbibresource{bib/thesis.bib}
% \addbibresource{bib/chap2.bib}

\begin{document}

%% 无编号内容:中英文论文封面、授权页
%# -*- coding: utf-8-unix -*-
\title{空气质量传感器管理系统设计与实现}
\author{孙\quad{}逢}
\advisor{吴刚副教授}
 \coadvisor{施宇晨}
\defenddate{2016年6月10日}
\school{上海交通大学}
\institute{电子信息与电气工程学院软件学院}
\studentnumber{5120309032}
\major{软件工程专业}

\englishtitle{Design and Implementation of Air Quality Sensor Management System}
\englishauthor{\textsc{Feng Sun}}
\englishadvisor{Prof. \textsc{Gang Wu}}
% \englishcoadvisor{Prof. \textsc{Uom Uom}}
\englishschool{Shanghai Jiao Tong University}
\englishinstitute{\textsc{Depart of XXX, School of Software} \\
  \textsc{Shanghai Jiao Tong University} \\
  \textsc{Shanghai, P.R.China}}
\englishmajor{Software Engeering}
\englishdate{Jun. 10th, 2016}


\maketitle

%% \makeenglishtitle

\makeatletter
\ifsjtu@submit\relax
	\includepdf{pdf/original.pdf}
	\cleardoublepage
	\includepdf{pdf/authorization.pdf}
	\cleardoublepage
\else
	\makeDeclareOriginal
	\makeDeclareAuthorization
\fi
\makeatother


\frontmatter 	% 使用罗马数字对前言编号

%% 摘要
\pagestyle{main}
%# -*- coding: utf-8-unix -*-
%%==================================================
%% abstract.tex for SJTU Master Thesis
%%==================================================

\begin{abstract}

近几年,空气质量问题已经逐渐成为中国乃至全世界的社会焦点,但大多数地区只有政府部门会有监测站和监测点,普通民众并不能实时了解自己身边的空气质量。Clarity Movement Co. 研发出了火柴盒大小但精度可媲美光学空气质量分析仪的PM2.5传感器,并通过与企业和政府合作在传统硬件设备和城市设施上搭载传感器来销售传感器。公司为之搭建了功能强大的云服务平台,可以实时地处理和推送空气质量数据,但还需要一个信息系统来操作和管理传感器以及展示云服务平台的强大。

在此背景下,本课题调研了大量JS框架和库基于现代主流的WEB开发技术为之开发了一个传感器管理系统。系统包含三个模块,分别是版本管理、Smart Home和Smart City模块。版本管理模块面向公司的硬件开发人员,让他们能够方便地管理自己开发的传感器及其版本、批次、兼容性和拥有者。Smart Home模块面向家电制造业的合作企业,让其能够使自己的家电互相联系起来。Smart City模块面向政府合作者,让其能够直观地从不同角度查看和下载城市空气质量数据。

\keywords{\large 空气质量 \quad 传感器 \quad 信息系统\quad WEB开发}
\end{abstract}

\begin{englishabstract}

In recent years, air quality has gradually became a social focus of China and even the world. However, in most regional only governments have monitoring stations and sites and ordinary people can not get real-time air quality nearby. Clarity Movement Co. developed matchbox-sized PM2.5 air quality sensors which can rival the optical analyzer in accuracy and sell them by mounting them in the traditional hardware and city facilities in cooperation with businesses and governments. The company has built a powerful cloud service platform which can handle and push real-time air quality data but still an information system is required to manage and operate the sensor and show the power of the platform.

In this context, this paper investigates a large number of JS frameworks and libraries to develop a sensor management system based on modern mainstream web-development technology. The system consists of three modules, namely, Version Management, Smart Home and Smart City. Version Management module is for the company's hardware developers to easily manage the sensors' versions, batches, compatibilities and users. Smart Home module is for cooperating enterprises who are in appliance manufacturing to be able to link their appliances to each other. Smart City module is for cooperating governments to visually view and download air quality of the city from different angles.

\englishkeywords{\large Air quality, Sensor, Information System, Web-development}
\end{englishabstract}



%% 目录、插图目录、表格目录
\tableofcontents
\listoffigures
\addcontentsline{toc}{chapter}{\listfigurename} %将插图目录加入全文目录
\listoftables
\addcontentsline{toc}{chapter}{\listtablename}  %将表格目录加入全文目录
% \listofalgorithms
% \addcontentsline{toc}{chapter}{算法索引}        %将算法目录加入全文目录


\mainmatter	% 使用阿拉伯数字对正文编号

%% 正文内容
\pagestyle{main}
%# -*- coding: utf-8-unix -*-
%%==================================================
%% chapter01.tex
%%==================================================

%\bibliographystyle{sjtu2}%[此处用于每章都生产参考文献]
\chapter{绪论}
\label{chap:intro}
\section{课题背景和意义}
随着“雾霾”、“PM2.5”等话题的升温,空气质量问题逐渐成为人们关注的焦点,出门不仅要看“天气”还要看“空气”。虽然如今各式各样的空气质量网站、APP层出不穷,有实时的也有预测的,但大多数据来源于政府环境部门监测站,少部分来自网站自己搭建的监测点。然而空气质量与天气不同之处在于,它受时间和空间的变化影响更大,对人们健康的影响也更大。因此,如何满足个人用户对周边空气质量的实时掌控成为一大难题。

Clarity Movement Co.(以下简称Clarity)是一家研发“世界上最小的空气质量传感器”的创业公司,目前完成研发的第二代传感器尺寸与火柴盒差不多大,便携度大增的同时,精度依旧不弱于光学空气质量分析仪。Clairty的主要销售途径是与相关企业和政府合作,将传感器搭载在传统硬件设备和城市建筑上。Clairty现在依旧组建了自己的云服务平台,传感器通过手机蓝牙或者Wi-fi上传空气质量数据,服务器处理并保存相关数据。然而,要让该产品能够为大众所用,配套的软件系统还未完备。

随着业务的发展,一方面Clarity自己内部需要一个信息系统来管理自己生产的传感器和传感器的版本、批次、拥有者等信息,另一方面Clarity的合作方往往是政府部门和传统硬件厂商如空调、汽车行业的企业,他们一般都不具备很强的软件开发能力,所以Clarity需要为他们定制合适的传感器管理系统来查看空气质量数据和管理传感设备。因此,本课题研究的“空气质量传感器管理系统”(以下简称本系统)应运而生。

\section{项目定义}
本系统应当是一个全栈JS的WEB应用,具有实时性、现代化、单页面、响应式、模块化等特点,用于满足Clarity内部信息化需求和日益增长的业务发展的需要。

虽然总体上来讲是传感器管理系统,但需求中的确有一些互相关系不大的部分,所以需要分为互相关联但相对独立的模块。首先对于内部的传感器管理需求,主要是对版本的管理,本系统把它定义为“版本管理模块”,面向的用户是Clarity的硬件开发人员。其次对于合作方的需求,又可分为家庭和个人级别的“Smart Home模块”和城市级别的“Smart City模块”。Smart Home模块面向想要构建智慧家庭的传统家电厂商,目前的主要目标客户是日本一家家电制造企业,最终用户则是使用家电和智能家居的个人;Smart Home模块面向想要构建智慧城市的汽车制造商或者政府,目前的主要目标客户是美国一家计划做智慧城市的政府部门,最终用户是政府工作人员。

\section{主要研究内容}
本课题的任务是设计与实现一个拥有3个模块的设备管理系统,为完成该任务,进行了如下研究:
\begin{enumerate}
  \item 根据开发团队的技术储备和开发能力,结合公司的企业文化和经济水平,调研相关的理论、技术和工具,对于某些没有接触过或不熟悉的技术进行必要的学习和提高,如ReactJS就是本文作者在开发过程中自学的。保证在拥有充分的理论指导、充足的技术储备和高效的开发工具的情况下完成系统的设计与实现。
  \item 详细分析3个模块的需求,与需求方积极沟通,深入挖掘甚至需求方自己都没想过的需求,给出一系列对后续开发有帮助的文档、图表和设计;开发过程中频繁部署,及时地获得需求方的意见和反馈,及时地修复bug和调整需求方不满意的地方。
  \item 设计并坚决执行测试方案,测试过程与开发过程同步甚至更早,单元测试和人工测试相结合,同时保障软件局部代码健壮性和整体功能的实现。
\end{enumerate}

\section{论文组织结构}
本文一共分为八章,各章节介绍如下
\begin{description}
    \item[第一章] 绪论~~简单阐述了本课题研究的背景和意义、项目定义、主要研究内容和本论文的组织结构。
    \item[第二章] 相关理论、技术和工具研究~~通过对比WEB设计开发的一些理论、技术和工具介绍了本系统主要使用的设计理论、技术架构和开发工具以及使用它们的原因。
    \item[第三章] 需求分析~~分析了本系统的功能目标、用户用例、性能要求和用户运行环境,并根据设计稿设计了快速原型和数据字典
    \item[第四章] 系统设计与开发~~介绍了总体上的设计思路,然后按模块介绍了设计和开发过程,最后介绍了几个主要组件的详细设计
    \item[第五章] 系统最终实现效果~~按不同模块和组件展示实现效果
    \item[第六章] 系统测试和部署~~介绍了本系统的开发、测试和运行环境以及持续集成的配置
    \item[第七章] 总结与展望~~总结了研究内容是否完成,思考了研究过程中的不足,并介绍了下一步的工作计划。
\end{description}


%# -*- coding: utf-8-unix -*-
%%==================================================
%% chapter08.tex
%%==================================================

%\bibliographystyle{sjtu2}%[此处用于每章都生产参考文献]
\chapter{总结与展望}
\label{chap:summary}

\section{工作总结}

\section{工作展望}



\appendix	% 使用英文字母对附录编号,重新定义附录中的公式、图图表编号样式
\renewcommand\theequation{\Alph{chapter}--\arabic{equation}}	
\renewcommand\thefigure{\Alph{chapter}--\arabic{figure}}
\renewcommand\thetable{\Alph{chapter}--\arabic{table}}
\renewcommand\thealgorithm{\Alph{chapter}--\arabic{algorithm}}

%% 附录内容,本科学位论文可以用翻译的文献替代。

\backmatter	% 文后无编号部分

%% 参考资料
\printbibliography[heading=bibintoc]

%% 致谢、发表论文、申请专利、参与项目、简历
%% 用于盲审的论文需隐去致谢、发表论文、申请专利、参与的项目
\makeatletter

% 致谢页
%# -*- coding: utf-8-unix -*-
\begin{thanks}

  感谢悉心指导论文设计和撰写的指导老师吴刚!

  感谢提携我加入公司并在工作中给出指导和帮助的同事和同学施宇晨!

  感谢在论文写作期间配合延后新需求的CEO陆邓君!

  感谢在工作中通力合作的同事和同学傅浩南!

\end{thanks}
 	  %% 致谢

\makeatother

\end{document}
