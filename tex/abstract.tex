%# -*- coding: utf-8-unix -*-
%%==================================================
%% abstract.tex for SJTU Master Thesis
%%==================================================

\begin{abstract}

近几年,空气质量问题已经逐渐成为中国乃至全世界的社会焦点,但大多数地区只有政府部门会有监测站和监测点,普通民众并不能实时了解自己身边的空气质量。Clarity Movement Co. 研发出了火柴盒大小但精度可媲美光学空气质量分析仪的PM2.5传感器,并通过与企业和政府合作在传统硬件设备和城市设施上搭载传感器来销售传感器。公司为之搭建了功能强大的云服务平台,可以实时地处理和推送空气质量数据,但还需要一个信息系统来操作和管理传感器以及展示云服务平台的强大。

在此背景下,本课题调研了大量JS框架和库基于现代主流的WEB开发技术为之开发了一个传感器管理系统。系统包含三个模块,分别是版本管理、Smart Home和Smart City模块。版本管理模块面向公司的硬件开发人员,让他们能够方便地管理自己开发的传感器及其版本、批次、兼容性和拥有者。Smart Home模块面向家电制造业的合作企业,让其能够使自己的家电互相联系起来。Smart City模块面向政府合作者,让其能够直观地从不同角度查看和下载城市空气质量数据。

\keywords{\large 空气质量 \quad 传感器 \quad 信息系统\quad WEB开发}
\end{abstract}

\begin{englishabstract}

In recent years, air quality has gradually become a social focus of China and even the world. However, in most regional only governments have monitoring stations and sites and ordinary people can not get real-time air quality nearby. Clarity Movement Co. developed matchbox-sized PM2.5 air quality sensors which can rival the optical analyzer in accuracy and sell them by mounting them in the traditional hardware and city facilities in cooperation with businesses and governments. The company has built a powerful cloud service platform which can handle and push real-time air quality data but still an information system is required to manage and operate the sensor and show the power of the platform.

In this context, this paper investigates a large number of JS frameworks and libraries to develop a sensor management system based on modern mainstream web-development technology. The system consists of three modules, namely, Version Management, Smart Home and Smart City. Version Management module is for the company's hardware developers to easily manage the sensors' versions, batches, compatibilities and users. Smart Home module is for cooperating enterprises who are in appliance manufacturing to be able to link their appliances to each other. Smart City module is for cooperating governments to visually view and download air quality of the city from different angles.

\englishkeywords{\large Air quality, Sensor, Information System, Web-development}
\end{englishabstract}

