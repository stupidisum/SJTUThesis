
%# -*- coding: utf-8-unix -*-
%%==================================================
%% chapter08.tex
%%==================================================

%\bibliographystyle{sjtu2}%[此处用于每章都生产参考文献]
\chapter{总结与展望}
\label{chap:summary}
\section{工作总结}
本系统基本完成了所有需求,并且三个模块全部正式上线正在运行。

Smart Home模块是最先上线的,在Clarity和目标客户(那家日本家电企业)洽谈时,让对方实际看到了Clarity软件开发能力和云服务平台的强大从而达成了合作意向。在有这个系统之前,别人看不见摸不着,并不知道Clarity的云服务平台有什么作用。

智慧城市是Clarity当前的业务重点,美国的那个城市政府是Clarity非常重要的一个合作伙伴,在Clarity对智慧城市这个需求的摸索过程中,Smart City模块经过了多次大的需求变动,最终还是第二个上线的,提供了很强大的数据可视化功能和下载功能。

版本管理模块的需求优先级在最初的规划中是最低的,所以它是最后一个开始开发的系统,但正所谓“磨刀不误砍柴工”,虽然代码生成器的开发额外占用了一些精力,但其开发周期反而是最短的。正在接收用户的反馈并且不断地改进之中。
\section{个人总结}
本文作者在一边开发一边学习ReactJS和Redux的过程中,不断地更新对软件开发和这两项技术的认知,不断地发现别人的更好的或最佳的实践(Best Practice),不仅给了我个人技术的成长,更是让我对这个时代的软件行业和JavaScript的世界更加热爱。

版本管理模块给了我一个巨大的惊喜,那就是虽然在给公司做私人项目,但还是有机会为开源社区贡献代码,并且那就是工作的一部分,让我也成为了开源世界的一个贡献者。
\section{工作展望}
从Clarity当前的业务重点来看,目前Smart Home模块的需求基本被搁置了。

主力需求是Smart City模块,这个模块已经迎来第三次大的需求和界面变动,是本团队下一步的工作重点。

版本管理模块还有很多让用户更加方便地增删改查的需求,如批量添加、复制添加、自动生成ID、上传文件等,也会逐步作为新特性添加到系统中。

在开源工作方面也有一些计划,比如持续改善generator-material-app这个代码生成器以及把客户端模型定义单独拿出来做一个开源项目,把这一生产力利器分享给世界。
