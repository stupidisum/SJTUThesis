%# -*- coding: utf-8-unix -*-
%%==================================================
%% chapter01.tex
%%==================================================

%\bibliographystyle{sjtu2}%[此处用于每章都生产参考文献]
\chapter{绪论}
\label{chap:intro}
\section{课题背景和研究意义}
随着“雾霾”、“PM2.5”等话题的升温,空气质量问题逐渐成为人们关注的焦点,出门不仅要看“天气”还要看“空气”。虽然如今各式各样的空气质量网站、APP层出不穷,有实时的也有预测的,但大多数据来源于政府环境部门监测站,少部分来自网站自己搭建的监测点。然而空气质量与天气不同之处在于,它受时间和空间的变化影响更大,对人们健康的影响也更大。因此,如何满足个人用户对周边空气质量的实时掌控成为一大难题。

Clarity Movement Co.(以下简称Clarity)是一家研发“世界上最小的空气质量传感器”的创业公司,目前完成研发的第二代传感器尺寸与火柴盒差不多大,便携度大增的同时,精度依旧不弱于光学空气质量分析仪。Clairty的主要销售途径是与相关企业和政府合作,将传感器搭载在传统硬件设备和城市建筑上。Clairty现在依旧组建了自己的云服务平台,传感器通过手机蓝牙或者Wi-fi上传空气质量数据,服务器处理并保存相关数据。然而,要让该产品能够为大众所用,配套的软件系统还未完备。

随着业务的发展,一方面Clarity自己内部需要一个信息系统来管理自己生产的传感器和传感器的版本、批次、拥有者等信息,另一方面Clarity的合作方往往是政府部门和传统硬件厂商如空调、汽车行业的企业,他们一般都不具备很强的软件开发能力,所以Clarity需要为他们定制合适的传感器管理系统来查看空气质量数据和管理传感设备。因此,本课题研究的“空气质量传感器管理系统”(以下简称本系统)应运而生。

\section{课题主要研究内容}
本系统分3个模块,版本管理模块、Smart Home模块和Smart City模块。
\subsection{版本管理模块}
本模块的名字叫“Balanar”\footnote{此名字本身无实际含义,只作代号},用户是Clarity的硬件开发人员。原本用户使用Excel记录传感器版本、批次、拥有者等信息,但随着传感器的不断改进,合作方越来越多,制作给合作方的特殊版本越来越多,零件进货的批次各有不同,使用Excel的时间成本和精力成本越来越高。所以Balanar解决的问题和实现的功能如下:
\begin{enumerate}
  \item 版本、批次和传感器之间的关系混乱,包括硬件版本、固件版本、软件版本、五种零件版本和五种零件的批次;Balanar要理清关联关系,用最佳的数据结构体现;
  \item 版本编号格式难以维护;Balanar要能够自动校验格式并给出错误描述;
  \item 硬件版本和固件版本兼容,固件版本又和软件版本兼容;Balanar要能够记录和判断版本兼容性
\end{enumerate}

\subsection{Smart Home模块}
本模块的名字叫“Robotic”\footnote{此名字含义为机器人的,因为robot太短太普遍},目标客户是日本一家家电制造企业,最终用户则是使用家电和智能家居的个人。该企业生产的产品包括空调和空气净化器,因此Clarity为其量身打造了把两者相结合的Smart Home。
空气净化器只能一直开着或者在人的操控下间歇性地净化空气,而搭载空气质量传感器的空调也只能检测空气质量并报告给用户,并不能实际改善空气质量,于是Robotic引入了一个可编程的扫地机器人完美地弥补了两者的不足。Robotic按以下流程工作:
\begin{enumerate}
  \item 不同房间的空气质量传感器持续上传数据到服务器,服务器保存数据;
  \item 服务器定时地计算并评估空气质量的好坏;
  \item 一旦有房间空气质量低于一定水平,服务器就会给机器人下达净化指令,机器人会移动到对应的房间并开启空气净化器;
  \item 空气质量好转之后,服务器再给机器人下达停止净化指令,机器人会关掉空气净化器原地待命;
  \item 机器人有手动模式,用户可以开启手动模式并直接指挥其到某房间净化;
\end{enumerate}

另外用户还可以查看个人携带的传感器空气质量、城市的空气质量以及每个房间的空气质量折线图。

\subsection{Smart City模块}
本模块的名字叫“Azwraith”\footnote{此名字本身无实际含义,只作代号},目标客户是美国一家计划做智慧城市的政府部门,最终用户是政府工作人员。该城市计划大量安装Clarity的传感器以监控城市空气质量,Azwraith为之提供了配套功能如下:
\begin{enumerate}
  \item 城市会安装和拆除传感器;Azwraith需要能够添加和删除设备(传感器);
  \item 政府工作人员可以地图形式直观的查看传感器位置和当前空气质量;
  \item 政府工作人员可以图表形式直观地查看并下载传感器最近历史数据,并且可以选择不同时间精度和空气质量度量;
  \item 政府工作人员可以表单形式选择不同时间跨度和精度来下载设备的长期历史数据;
  \item 政府工作人员可以控制设备在地图和图表上是否显示;
\end{enumerate}


\section{论文组织结构}
本文一共分为八章,各章节介绍如下
\begin{description}
    \item[第一章] 绪论~~简单阐述了本课题研究的背景、意义和主要内容以及本论文的组织结构。
    \item[第二章] WEB设计开发理论、技术和工具~~通过对比WEB设计开发的一些理论、技术和工具介绍了本系统主要使用的设计理论、技术架构和开发工具以及使用它们的原因。
    \item[第三章] 可行性分析~~分析了本系统的技术可行性、经济可行性和操作可行性。
    \item[第四章] 需求分析~~分析了本系统的功能目标、用户用例、性能要求和用户运行环境,并根据设计稿设计了快速原型和数据字典
    \item[第五章] 系统设计与开发~~介绍了总体上的设计思路,然后按模块介绍了设计和开发过程,最后介绍了几个主要组件的详细设计
    \item[第六章] 系统最终实现效果~~按不同模块和组件展示实现效果
    \item[第七章] 系统测试和部署~~介绍了本系统的开发、测试和运行环境以及持续集成的配置
    \item[第八章] 总结与展望~~总结了研究内容是否完成,思考了研究过程中的不足,并介绍了下一步的工作计划。
\end{description}
