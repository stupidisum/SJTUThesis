%# -*- coding: utf-8-unix -*-
%%==================================================
%% chapter01.tex
%%==================================================

%\bibliographystyle{sjtu2}%[此处用于每章都生产参考文献]
\chapter{绪论}
\label{chap:intro}
\section{课题背景和意义}
随着“雾霾”、“PM2.5”等话题的升温,空气质量问题逐渐成为人们关注的焦点,出门不仅要看“天气”还要看“空气”。虽然如今各式各样的空气质量网站、APP层出不穷,有实时的也有预测的,但大多数据来源于政府环境部门监测站,少部分来自网站自己搭建的监测点。然而空气质量与天气不同之处在于,它受时间和空间的变化影响更大,对人们健康的影响也更大。因此,如何满足个人用户对周边空气质量的实时掌控成为一大难题。

Clarity Movement Co.(以下简称Clarity)是一家研发“世界上最小的空气质量传感器”的创业公司,目前完成研发的第二代传感器尺寸与火柴盒差不多大,便携度大增的同时,精度依旧不弱于光学空气质量分析仪。Clairty的主要销售途径是与相关企业和政府合作,将传感器搭载在传统硬件设备和城市设施上。Clairty现在已经组建了自己的云服务平台,传感器通过手机蓝牙或者Wi-fi上传空气质量数据,服务器处理并保存相关数据。然而,要让该产品能够为大众所用,配套的软件系统还未完备。

随着业务的发展,一方面Clarity自己内部需要一个信息系统来管理自己生产的传感器和传感器的版本、批次、拥有者等信息,另一方面Clarity的合作方往往是政府部门和传统硬件厂商如空调、汽车行业的企业,他们一般都不具备很强的软件开发能力,所以Clarity需要为他们定制合适的传感器管理系统来查看空气质量数据和管理传感设备。因此,本课题研究的“空气质量传感器管理系统”(以下简称本系统)应运而生。

\section{项目定义}
本系统应当是一个全栈JS的WEB应用,具有实时性、现代化、单页面、响应式、模块化等特点,用于满足Clarity内部信息化需求和日益增长的业务发展的需要。

虽然总体上来讲是传感器管理系统,但需求的确可以被划分为一些互相关系不大的部分,为了保证系统的可扩展性和低耦合性,本系统需要分为互相关联但相对独立的模块。首先对于内部的传感器管理需求,主要是对版本的管理,本系统把它定义为“版本管理模块”,面向的用户是Clarity的硬件开发人员。其次对于合作方的需求,又可分为家庭和个人级别的“Smart Home模块”和城市级别的“Smart City模块”。Smart Home模块面向想要构建智慧家庭的传统家电厂商,目前的主要目标客户是日本一家家电制造企业,最终用户则是使用家电和智能家居的个人;Smart Home模块面向想要构建智慧城市的汽车制造商或者政府,目前的主要目标客户是美国一家计划做智慧城市的政府部门,最终用户是政府工作人员。

\section{主要研究内容}
本课题旨在确保本系统的设计和实现在有充分的理论指导、充足的技术储备、高效的开发工具、详细的需求分析、及时的用户反馈和完备的测试方案的情况下完成。

本课题首先根据开发团队的技术储备和开发能力,结合公司的企业文化和经济水平,调研相关的理论、技术和工具,对于某些没有接触过或不熟悉的技术进行必要的学习和提高,如ReactJS就是本文作者在开发过程中自学的。其次详细分析需求,与需求方积极沟通,深入挖掘需求一些甚至需求方自己都没想过的需求,给出一系列对后续开发有帮助的文档、图表和设计。然后开发过程中频繁部署,及时地获得需求方的意见和反馈,及时地修复bug和调整需求方不满意的地方。最后设计并坚决执行测试方案,测试过程与开发过程同步甚至更早,单元测试和人工测试相结合,同时保障软件局部代码健壮性和整体功能的实现。

\section{论文组织结构}
本文一共分为八章,各章节介绍如下
\begin{description}
    \item[第一章] 绪论~~简单阐述了本课题研究的背景和意义、项目定义、主要研究内容和本论文的组织结构。
    \item[第二章] 相关理论、技术和工具研究~~通过对比WEB设计开发的一些理论、技术和工具介绍了本系统主要使用的设计理论、技术架构和开发工具以及使用它们的原因。
    \item[第三章] 需求分析~~分析了本系统的功能目标、用户用例、性能要求和用户运行环境,并根据设计稿设计了快速原型和数据字典
    \item[第四章] 系统设计与开发~~介绍了总体上的设计思路,然后按模块介绍了设计和开发过程,最后介绍了几个主要组件的详细设计
    \item[第五章] 系统最终实现效果~~按不同模块和组件展示实现效果
    \item[第六章] 系统测试和部署~~介绍了本系统的开发、测试和运行环境以及持续集成的配置
    \item[第七章] 总结与展望~~总结了研究内容是否完成,思考了研究过程中的不足,并介绍了下一步的工作计划。
\end{description}
