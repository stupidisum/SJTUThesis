% 无英文大摘要时的字符数
% 词数    : 30354
% 字符数  : 48537
% 中文字数: 26995

\chapter{DESIGN AND IMPLEMENTATION OF AIR QUALITY SENSOR MANAGEMENT SYSTEM}

In recent years, with the heating topics like haze and PM2.5, air quality has gradually become a social focus of China and even the world. Now gentlemen have to talk about in daily chat not only the weather but also the air. Although there is an endless stream of historic, real-time, forecast air quality websites and applications nowadays, almost all of them get air quality data from monitoring stations of government environmental department and only a few of them built their own monitoring sites. However, air quality is more complex than weather. Time and space have greater impact on it than weather. And it has greater impact on people's health than weather. Therefore, how to meet the individuals' willing to grasp the real-time air quality nearby has become a major problem.

'Clarity Movement Co.', hereinafter referred to as Clarity, is a start-up company developed the "world's smallest air quality sensor". The current second-generation sensor is about the same size as a matchbox which has extremely increased its portability while still has accuracy not weaker than the optical air quality analyzer. The main sales channel of Clairty is to cooperate with the relevant business and government and to mount the sensor on conventional hardware devices and urban facilities. Now, Clairty have built up their own cloud services platform on which sensors upload air quality data through WiFi or smart phone's bluetooth and the data can be stored, processed, and pushed to any client by the server. However, to show how powerful the cloud is, Clarity still need an information system to control and manage sensors including view and download air quality data.

On the one hand, with the development of Clarity's business, Clarity's internal staff need an information system to manage their sensors produced by them and sensors' information, like versions, batches and owners. On the other hand, Clarity's cooperators, who tend to be government departments and traditional hardware manufacturers such as companies in industry of home appliance and automotive, generally do not have strong capabilities of software development, so Clarity need to customize appropriate sensor management system for them to view air quality data and manageme sensing devices. Thus, what this paper researches, Air Quality Sensor Management System, hereinafter referred to as this system, came into being. This system should be a modern Full-Stack JavaScript WEB application to meet the requirements of Clarity's informatization and the needs of growing-up business with features like real-time, single-page, responsive, modular, etc.

This study aims to ensure the design and implementation of this system is under the circumstance of adequate theoretical guidance, sufficient technical reserves, efficient development tools, detailed requirement analysis, timely user feedback and comprehensive test plan. This study firstly researches related theories, techniques and tools and learns or improves knowledge of some never-learned or unfamiliar ones of them, such as ReactJS, which the author has never touched and self-learns during the process of developing according to development team's technical reserves and development capabilities, combined with the company's corporate culture and economic level. Secondly, this study analyses the requirements exhaustively and sums up a series documents, graphics and design which will be helpful in the following development by communicating with the demand side actively and dig the requirements deeply, some even the demand side themselves never thought they would need. Thirdly, this study deploys frequently during development process, timely access to suggestions and feedbacks on the demand side, timely fixes bugs and refactor features that requirements are dissatisfied with. Finally, design and resolutely implement testing plans, which should start synchronously with or even earlier than the development, including unit tests and e2e tests safeguarding software robustness of the codes locally and functionality of the application globally.

Although generally speaking this system is a sensor management system, the demand can indeed be divided into a number of parts with little relationship with each other, so in order to ensure this system's scalability and low coupling, it needs to be divided into several interrelated but relatively independent modules. For the internal sensor management needs, primarily for the management of the versions, this system defines it as a module named 'Version Management'. For the demands for partners, this system defines the family and individual levels part as ‘Smart Home’ and the city level part as ‘Smart City’.

Version management is facing Clarity's hardware developers who used to use Excel to record sensor's versions, batches, owners and other information. However, with the continuous improvement of the sensor and purchase of the components, there are more and more versions, component batches, partners and special release for partners, using Excel costs users more and more time and energy. The system needs to be able to allow users to deal with the CRUD operations of and the complex relationships among sensors, users, compatibilities, hardware versions, firmware versions, software versions and version and batches of five kinds of components. And this system needs to process some special format check, so when the users input they can timely get validations and receive feedbacks.

Smart Home is for traditional home appliance manufacturers who want to build smart home. The current main target client is a Japanese appliance manufacturer and the end-users are the individuals using those smart home appliances. Common air purifiers can only be purifying all day or intermittently clean the air under human control and appliances equipped with air quality sensors can only monitor air quality and report to the user but not actually improve air condition. Clarity introduced a programmable intelligent sweeping robot between them to cleverly compensate for the shortcomings of both so that the robot moves to different rooms where the sensors are located and turn on the air purifier according to instructions from the server, which detects changes of the rooms' air quality in real time to determine instructions to the robot. Users can view the robot's position and working status on WEB application and control it manually, as well as view the current air condition and history of air quality in different rooms.

Smart City's target clients are manufacturers and governments want to build smart city. Currently the main target client is a city government of the United States planning to build a smart city and the end users are government employees. Users want to be able to add, delete, show and hide devices (sensors) applications on the WEB. This system can visually and intuitively display the current air quality and position of devices in the form of map, show real-time and latest historical data of some specificated equipment graphically in chart, support downloading historical data for different devices, time spans, time precision and air quality measurements in csv form.

The important business logic of this system is covered by unit tests, the entire application is under end to end tests, the production branch, aka master branch, is configured to be automatic deployed using AWS's Code Pipeline service and other branches are using Solano CI for continuous integration.

All the three modules of this system are formally released and running online.

Smart Home is the first module online, in the negotiations between Clarity and a target client, that Japanese home appliance business mentioned above, the client actually saw Clarity's software development capabilities and the powerful cloud services platform and thus reached a cooperation. Before this module is online, others could not see or touch the cloud services platform so that didn't know its capability.

Smart city is the business focus of Clarity and that American city government is a very important partner to Clarity at present stage. In the process of Clarity's exploration of smart city, Smart City has undergone several big requirement changes, and ultimately becomes the second module gets online, providing very powerful capabilities of data visualization and data export.

Requirement priority of Version Management was the lowest in the initial planning, so it was the last one to start developing the system. As so-called 'Preparation may quicken the workers' says, although the development of additional code generator takes up some of the energy, instead Version Management has the shortest development cycle. It's receiving users' feedbacks and constantly being improved.

I, the author the this paper, in the side of the development process while self-learning ReactJS and some other technologies, constantly updated awareness of software development and those technologies, found better or best practices of others. The experience let me not only grow up technologically but also find myself love more the world of software industry and JavaScript in this era. Version Management module gave me a big surprise, that although I'm doing private projects for the company, but still have the opportunity to contribute code to the open source community, and that is part of the work, and let me became a formal contributor to the open source world.
